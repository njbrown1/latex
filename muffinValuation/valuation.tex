\documentclass[11pt]{article}
\usepackage{graphicx}
\usepackage{amsmath}
\usepackage{booktabs}
\usepackage{geometry}
\usepackage{hyperref}
\usepackage{float}
\usepackage{setspace}  % For spacing control, optional
\geometry{margin=1in}
\setlength{\parindent}{0pt}

\begin{document}

% Simulated Title at Top of First Page
\begin{center}
    \vspace*{0.5cm}
    {\Huge\bfseries Valuation Case Study\par}
    \vspace{0.8cm}
    {\LARGE\itshape Brad's Bakery\par}
    \vspace{0.8cm}
    {\large Nathan Brown\par}
    {\large \today\par}
    \vspace{1cm}
\end{center}

\tableofcontents
\newpage

\section{Company Overview}
Brad’s Bakery, founded in 2020, is a U.S.-based baked goods company that operates primarily in the Pacific Northwest. The company sells a variety of high-quality bakery items—including pastries, artisanal breads, and other baked treats—through a franchise model. Brad’s differentiates itself by focusing on premium, health-conscious ingredients, allowing it to position its products at a higher price point than many competitors. \\

While the bakery industry is highly competitive, Brad’s Bakery has built a loyal customer base by combining product quality with strong, targeted marketing. The brand does not offer any services—its focus is entirely on its product line, which supports consistency and scalability across franchise locations.

\section{Financial Analysis}
Brad's Bakery has strong recurring revenue which has grown 20\% YoY over the past 5 years, due to their marketing gaining increased traction among young (18-30 year-old), health-conscious adults. Brad's has a Gross Margin of 70\%, which is excellent compared to the industry average of 40-60\%. Brad's also maintains constant operating expenses. This is due to new dough refrigeration technology and Brad's use of artificial intelligence (AI) to generate advertising material, which allows them to keep employee headcount low. See Table 1 for a summary of Brad's income statement:
\begin{table}[H]
    \centering
    \caption{Income Statement Summary (in \$ millions)}
    \begin{tabular}{lrrrrr}
        \toprule
        & 2021 & 2022 & 2023 & 2024 & 2025 \\
        \midrule
        Revenue                  & 100.0 & 120.0 & 144.0 & 172.8 & 207.4 \\
        Cost of Goods Sold       & 30.0 & 36.0 & 43.2 & 51.8 & 62.2 \\
        Operating Expenses       & 10.0 & 10.0 & 10.0 & 10.0 & 10.0 \\
        \addlinespace 
        Gross Profit             & 70.0 & 94.0 & 100.8 & 121.0 & 145.2 \\ 
        Net Income               & 40.0 & 51.2 & 64.7 & 80.8 & 100.0 \\
        \bottomrule
    \end{tabular}
\end{table}
Brad's has solid cashflow and has not needed to purchase new PP\&E since its inception. See Table 2 for a summary of Brad's cash flow statement:
\begin{table}[H]
    \centering
    \caption{Cash Flow Statement Summary (in \$ millions)}
    \begin{tabular}{lrrrrr}
        \toprule
        & 2021 & 2022 & 2023 & 2024 & 2025 \\
        \midrule
        Cash from Operating Activities & 50.0 & 61.2 & 74.6 & 90.8 & 110.1 \\
        Cash from Investing Activities & (50.0) & 0.0 & 0.0 & 0.0 & 0.0 \\
        Cash from Financing Activities & 0.0 & 0.0 & 0.0 & 0.0 & 0.0 \\
        \addlinespace
        Net Change in Cash            & 0.0 & 61.2 & 74.6 & 90.8 & 110.1 \\
        \bottomrule
    \end{tabular}
\end{table}
\section{Estimated Valuation}
\section{Investment Recommendation}
We recommend that \textbf{\$700-900M} as an appropriate price to buy Brad's bakery.
\end{document}
