\documentclass[11pt]{article}
\usepackage{graphicx}
\usepackage{amsmath}
\usepackage{booktabs}
\usepackage{geometry}
\usepackage{hyperref}
\usepackage{float}
\usepackage{setspace}
\geometry{margin=1in}
\setlength{\parindent}{0pt}

\begin{document}

\begin{center}
    \vspace*{0.5cm}
    {\Huge\bfseries Valuation Case Study\par}
    \vspace{0.8cm}
    {\LARGE\itshape Brad's Bakery\par}
    \vspace{0.8cm}
    {\large Nathan Brown\par}
    {\large \today\par}
    \vspace{1cm}
\end{center}

\tableofcontents
\newpage

\section{Company Overview}
    \subsection{Company Description}
        Brad’s Bakery, founded in 2020, is a U.S.-based baked goods company that operates primarily in the Pacific Northwest. The company sells a variety of high-quality bakery items—including pastries, artisanal breads, and other baked treats—through a franchise model. Brad’s differentiates itself by focusing on premium, health-conscious ingredients, allowing it to position its products at a higher price point than many competitors.
    \subsection{Market Placement}
        While the bakery industry is highly competitive, Brad’s Bakery has built a loyal customer base by combining product quality with strong, targeted marketing. The brand does not offer any services—its focus is entirely on its product line, which supports consistency and scalability across franchise locations.

\section{Financial Analysis}
    Brad's Bakery has strong recurring revenue which has grown 20\% YoY over the past 5 years, due to their marketing gaining increased traction among 18-30 year-old, health-conscious adults. Brad's has a Gross Margin of 70\%, which is excellent compared to the industry average of 40-60\%. Brad's also maintains constant operating expenses. This is due to new dough refrigeration technology and Brad's use of artificial intelligence (AI) to generate advertising material, which allows them to keep employee headcount low. See Table 1 for a summary of Brad's income statement:
    \begin{table}[H]
        \centering
        \caption{Income Statement Summary (in \$ millions)}
        \begin{tabular}{lrrrrr}
            \toprule
            & 2021 & 2022 & 2023 & 2024 & 2025 \\
            \midrule
            Revenue                  & 100.0 & 120.0 & 144.0 & 172.8 & 207.4 \\
            Cost of Goods Sold       & 30.0 & 36.0 & 43.2 & 51.8 & 62.2 \\
            Operating Expenses       & 10.0 & 10.0 & 10.0 & 10.0 & 10.0 \\
            EBITDA                   & 60.0 & 74.0 & 90.8 & 111.0 & 135.2 \\
            \addlinespace 
            Gross Profit             & 70.0 & 94.0 & 100.8 & 121.0 & 145.2 \\ 
            Net Income               & 40.0 & 51.2 & 64.7 & 80.8 & 100.0 \\
            \bottomrule
        \end{tabular}
    \end{table}
    Brad's has increasing cashflow and has not needed to purchase new PP\&E since its inception. However, in 2026, Brad's plans to spend at least \$100m in new capital to replace their defunct PP\&E. See Table 2 for a summary of Brad's cash flow statement:
    \begin{table}[H]
        \centering
        \caption{Cash Flow Statement Summary (in \$ millions)}
        \begin{tabular}{lrrrrr}
            \toprule
            & 2021 & 2022 & 2023 & 2024 & 2025 \\
            \midrule
            Cash from Operating Activities & 50.0 & 61.2 & 74.6 & 90.8 & 110.1 \\
            Cash from Investing Activities & (50.0) & 0.0 & 0.0 & 0.0 & 0.0 \\
            Cash from Financing Activities & 0.0 & 0.0 & 0.0 & 0.0 & 0.0 \\
            \addlinespace
            Net Change in Cash            & 0.0 & 61.2 & 74.6 & 90.8 & 110.1 \\
            \bottomrule
        \end{tabular}
    \end{table}
    Brad's has not used any of its cash flow to invest in marketable securities or similar assets, instead opting to keep its assets as predominately cash. See Table 3 for a summary of Brad's balance sheet:
    \begin{table}[H]
        \centering
        \caption{Balance Sheet Summary (in \$ millions)}
        \begin{tabular}{lrrrrr}
            \toprule
            & 2021 & 2022 & 2023 & 2024 & 2025 \\
            \midrule
            \textbf{Assets} \\
            Cash \& Cash Equivalents     & 100.0 & 161.2 & 235.8 & 326.6 & 436.7 \\
            PP\&E (Net)                  & 40.0 & 30.0 & 20.0 & 10.0 & 0.0 \\
            \addlinespace
            Total Assets                 & 140.0 & 191.2 & 255.8 & 336.6 & 436.7 \\
            \midrule
            \textbf{Liabilities \& Shareholders' Equity} \\
            Common Stock                 & 100.0 & 100.0 & 100.0 & 100.0 & 100.0 \\
            Retained Earnings            & 40.0 & 91.2 & 155.8 & 236.6 & 336.7 \\
            \addlinespace
            Total Liabilities \& Equity & 140.0 & 191.2 & 255.8 & 336.6 & 436.7 \\
            \bottomrule
        \end{tabular}
    \end{table}
\section{Estimated Valuation}
    \subsection{Drivers of Net Income}
    We identify Brad's constant, low operating expenses as a key driver of net income. This allows Brad's to continually increase its EBITDA by 23\% YoY, which indicates that Brad's has strong growth potential. This is a major strength for Brad's relative to its competitors. \\\\
    We also consider Brad's cost of goods sold (COGS) to be extremely strong relative to its industry competitors. While Brad's buys expensive, healthier ingredients, they charge much higher prices and customers still purchase their products thanks to high brand-name recognition and quality of advertising. This helps keep gross margins \& net income high.
    \subsection{Discounted Cash Flow Analysis}
    We will first calculate the discount rate (WACC) for Brad's bakery:
    \[
        \text{WACC} = \left( \frac{E}{V} \times r_e \right) + \left( \frac{D}{V} \times r_d \times (1 - T) \right)
    .\]
    Because Brad's has no debt ($D$), $D = 0$ and $\tfrac{E}{V}$ = 1, and the equation simplifies drastically to
    \[
        \text{WACC} = \text{Cost of Equity} = r_e = r_f + \beta (r_m - r_f)
    .\] 
    The current risk-free rate ($r_f$) is 4.59\%\footnote{10-year treasury bond yield as of May 2025}. We set $\beta = 1.6$ to acknowledge that Brad's is more risky than the average business due to the high competition in the baking industry. We estimate the equity risk premium $(r_m - r_f)$ as 5.0\%. Thus
    \[
        \text{WACC} = 4.59\% + 1.6 \times 5.0\% = 12.6\%
    .\] 
    Relative to other businesses, this is a high discount rate, but we feel that it is warranted, due to the high competition in the industry and the fact that Brad's does not have a strong moat (aside from their customer loyalty). We also expect Brad's revenue growth to grow at a terminal growth rate of $r = 2.4$\%. Therefore, we can can calculate the approximate enterprise value (EV):
    \begin{align*}
        EV &= \sum_{t=1}^{5} \frac{\text{FCF}_t}{(1 + r_e)^t} + \frac{\text{TV}}{(1 + r_e)^5}
        \quad \text{where} \quad \text{TV} = \frac{\text{FCF}_6}{r_e - g} \\
           &= 44.2 + 47.9 + 51.7 + 55.7 + 59.8 + 577.3 = 836.6 
    \end{align*}
    This provides an estimate for the valuation of Brad's Bakery. This is equivalent to 6.2x EBITDA, which is low for a consumer goods company, and 4x revenue, which is high for a consumer goods company. Overall, the DCF, N$\times$Revenue, and N$\times$EBITDA valuation frameworks all generally agree on the valuation of Brad's.
\section{Investment Recommendation}
    \subsection{Executive Summary}
        We recommend \textbf{buying} Brad's Bakery at a price of \textbf{\$700-900M}, due to Brad's strong YoY growth, cash flow, and strength of branding. 
    \subsection{Risks}
        This is an investment with low-to-medium risk, given that companies in the consumer goods industry tend to be stable. However, we recommend considering the following risks:
        \begin{itemize}
            \item Brad's has no strong moat
            \item Brad's will have a strong CapEx spike in 2026
            \item Brad's revenue could be easily disrupted with new firms entering the industry
        \end{itemize}
    \subsection{Change the Management}
        If you do purchase Brad's Bakery, we recommend firing the CFO immediately and replacing him with someone who will actually use the cash to purchase new capital and marketable securities.\footnote{Keep the advertising people---they are doing great.}
\end{document}
